\documentclass[a4paper,12pt,titlepage,oneside]{book}
\linespread{1.2}
\pagestyle{plain}


\usepackage[italian]{babel} 
\usepackage{picture}
\usepackage{hyperref}
\usepackage[Symbol]{upgreek}
\usepackage{amsmath}
\usepackage{wrapfig}
\usepackage{siunitx}
\usepackage{caption}
\usepackage{subcaption}
\usepackage{graphicx}
\usepackage{csquotes}
\usepackage{color}
\usepackage{listings}
\usepackage{xcolor}

\definecolor{codegreen}{rgb}{0,0.6,0}
\definecolor{codegray}{rgb}{0.5,0.5,0.5}
\definecolor{codepurple}{rgb}{0.58,0,0.82}
\definecolor{backcolour}{rgb}{0.95,0.95,0.92}

\lstdefinestyle{mystyle}{
    backgroundcolor=\color{backcolour},   
    commentstyle=\color{codegreen},
    keywordstyle=\color{magenta},
    numberstyle=\tiny\color{codegray},
    stringstyle=\color{codepurple},
    basicstyle=\ttfamily\footnotesize,
    breakatwhitespace=false,         
    breaklines=true,                 
    captionpos=b,                    
    keepspaces=true,                 
    numbers=left,                    
    numbersep=5pt,                  
    showspaces=false,                
    showstringspaces=false,
    showtabs=false,                  
    tabsize=2
}

\lstset{style=mystyle}

\usepackage{pdfpages}
\textheight24cm\topmargin0mm\headheight0mm\headsep6mm\oddsidemargin20pt\evensidemargin30pt

\usepackage[T1]{fontenc}
\usepackage[utf8]{inputenc}
\usepackage{titlesec, blindtext, color}
\definecolor{gray75}{gray}{0.75}
\newcommand{\hsp}{\hspace{20pt}}
\titleformat{\chapter}[hang]{\Huge\bfseries}{\thechapter\hsp\textcolor{gray75}{|}\hsp}{0pt}{\Huge\bfseries}

\usepackage{csquotes}
\usepackage[backend=bibtex]{biblatex}
\bibliography{bibliografia}


\definecolor{dkgreen}{rgb}{0,0.6,0}
\definecolor{gray}{rgb}{0.5,0.5,0.5}
\definecolor{mauve}{rgb}{0.58,0,0.82}

\begin{document}

\begin{titlepage}

\newcommand{\HRule}{\rule{\linewidth}{0.5mm}} % Defines a new command for the horizontal lines, change thickness here

\center % Center everything on the page
 
%----------------------------------------------------------------------------------------
%	HEADING SECTIONS
%----------------------------------------------------------------------------------------

\textsc{\LARGE Università degli studi di Milano-Bicocca}\\[1cm] % Name of your university/college
\textsc{\Large Metodi del Calcolo Scientifico }\\[0.3cm] % Major heading such as course name
\textsc{\large Project 1}\\[0.1cm] % Minor heading such as course title

%----------------------------------------------------------------------------------------
%	TITLE SECTION
%----------------------------------------------------------------------------------------

\HRule \\[0.4cm]
{ \huge \bfseries Sistemi lineari con matrici sparse simmetriche e definite positive}\\[0.4cm] % Title of your document
\HRule \\[1.5cm]
 
%----------------------------------------------------------------------------------------
%	AUTHOR SECTION
%----------------------------------------------------------------------------------------

\large
\emph{Authors:}\\
Capra Riccardo - 808227 \\ % Your name
Curnis Giovanni - 807424 \\[1cm]

% If you don't want a supervisor, uncomment the two lines below and remove the section above
%\Large \emph{Author:}\\
%John \textsc{Smith}\\[3cm] % Your name

%----------------------------------------------------------------------------------------
%	DATE SECTION
%----------------------------------------------------------------------------------------

%{\large \today}\\[1cm] % Date, change the \today to a set date if you want to be precise

%----------------------------------------------------------------------------------------
%	LOGO SECTION
%----------------------------------------------------------------------------------------

\includegraphics[scale = 2.5]{src/image/unimib-logo.pdf}\\[1cm] % Include a department/university logo - this will require the graphicx package
 
%----------------------------------------------------------------------------------------

\vfill % Fill the rest of the page with whitespace

\end{titlepage}

\tableofcontents

\chapter{Introduzione}
Scopo di questo progetto era confrontare librerie Open source a Matlab in sistemi operativi diversi per la risoluzione di uno stesso problema, l'implementazione del metodo di Choleski per risolvere sistemi lineari per matrici sparse, simmetriche e definite positive.

Per risolvere i sistemi sono stati usati: \textbf{Matlab 2020a} ed \textbf{Octave 5.2}, inoltre per Linux è stata testata anche una libreria di \textbf{Python 3.6}.

Le tre metriche tenute in considerazione per il progetto sono state: il tempo necessario per risolvere il sistema, la quantità di memoria utilizzata e l'errore relativo.

    
\chapter{Cenni teorici}
\section{Matrice simmetrica}
Detta $A^T$ la matrice trasposta di $A$, una matrice $A$ è simmetrica~\cite{matsim} quando $A^T=A$.

\section{Matrice definita positiva}
In matematica, e più precisamente in algebra lineare, una matrice definita positiva~\cite{matpos} è una matrice quadrata $A$ tale che, detto  $x^*$  il trasposto complesso coniugato di $x$ , si verifica che la parte reale di $x^*Ax$ è positiva per ogni vettore complesso \begin{math} x \neq 0\end{math}.


\section{Matrice sparsa}
\textbf{Molti}, se non tutti, \textbf{valori di una matrice sparsa~\cite{matspa} sono pari a zero.} 
Il principio di base che si segue per memorizzare una matrice sparsa è di salvare, invece di tutti i valori, soltanto quelli diversi da zero. A seconda del numero e della distribuzione dei valori diversi da zero, si possono utilizzare diverse strutture dati ed ottenere risparmi considerevoli in termini di memoria impossibili da ottenere con l'approccio usuale.


\section{Decomposizione di Cholesky}
La decomposizione di Cholesky~\cite{disp} è un metodo di fattorizzazione matriciale semplice ed efficiente, applicabile a matrici generiche A ∈ R n×n che sono simmetriche e definite positive.

\textbf{Teorema}: Sia \begin{math}A \in R^{n \times n} \end{math} triangolare superiore tale che \begin{math} A = R^{t}R \end{math}.
La $R$ è unica se si aggiunge l’ulteriore richiesta che gli elementi sulla diagonale principale siano positivi.

La complessità computazionale della decomposizione di Cholesky è \begin{math} O(n^3) \end{math}.


\section{Decomposizione di Cholesky}
Dato un sistema lineare $Ax = b$, in cui la matrice $A$ gode delle proprietà necessarie per la decomposizione, è applicabile il metodo di Cholesky per risolverlo. Una volta ottenuta la matrice $R$ risulta immediata la risoluzione del sistema lineare come segue: si sostituisce alla matrice $A$ la sua fattorizzazione e si considera il sistema $R^tRx = b$; successivamente si risolvono in ordine un sistema triangolare inferiore e poi uno triangolare superiore:
\begin{itemize}
  \item $R^t y = b$
  \item $Rx = y$
\end{itemize}

I sistemi vengono risolti rispettivamente tramite sostituzione in avanti e all’indietro. In particolare, i sistemi lineari utilizzati nei solutori diretti per matrici simmetriche e definite positive sparse assumono la seguente forma: $Ax = b$ con $b$ scelto in modo tale che la soluzione esatta sia il vettore $x_e = [1 . . . 1]$ (quindi $b = Ax_e$ ).


\chapter{Impostazione del Progetto}
    \section{Matrici Usate per i Test}
    \noindent Nella fase di test sono state utilizzate le seguenti matrici sparse, simmetriche e definite positive:
    \begin{itemize}
        \item{\textbf{\href{https://sparse.tamu.edu/FIDAP/ex15}{ex15} :}} matrice di dimensioni 6867x6867 con un numero di elementi diversi da zero di 98671.

        \item{\textbf{\href{https://sparse.tamu.edu/MaxPlanck/shallow_water1}{shallow\_water1} :}} matrice di dimensioni 81920x81920 con un numero di elementi diversi da zero di 327680.
        
        \item{\textbf{\href{https://sparse.tamu.edu/Rothberg/cfd1}{cfd1} :}} matrice di dimensioni 70656x70656 con un numero di elementi diversi da zero di 1825580.
        
        \item{\textbf{\href{https://sparse.tamu.edu/Rothberg/cfd2}{cfd2} :}} matrice di dimensioni 123440x123440 con un numero di elementi diversi da zero di 3085406.
        
        \item{\textbf{\href{https://sparse.tamu.edu/Wissgott/parabolic_fem}{parabolic\_fem} :}} matrice di dimensioni 525825x525825 con un numero di elementi diversi da zero di 3674625.
        
        \item{\textbf{\href{https://sparse.tamu.edu/GHS_psdef/apache2}{apache2} :}} matrice di dimensioni 715176x715176 con un numero di elementi diversi da zero di 4817870.
        
        \item{\textbf{\href{https://sparse.tamu.edu/AMD/G3_circuit}{G3 circuit} :}} matrice di dimensioni 1585478x1585478 con un numero di elementi diversi da zero di 7660826.
    \end{itemize}
    \noindent Purtroppo le matrici \textit{\href{https://sparse.tamu.edu/Janna/Flan_1565}{Flan\_1565}} e \textit{\href{https://sparse.tamu.edu/Janna/StocF-1465}{StocF-1465}} non sono state incluse nell'analisi dei risultati in quanto non è stato possibile elaborarle con l'hardware a disposizione. La stima effettuata per l'esecuzione di queste matrici ha rilevato che sarebbero necessari più di 50GB di memoria per poterle elaborare e diverse ore di calcolo.
    
    \section{Librerie Utilizzate}
        \subsection{MATLAB}
        \noindent \textbf{MATLAB}~\cite{matlab} è un software proprietario della \href{https://www.mathworks.com}{MathWorks}, il nome è l'abbreviazione di \textbf{MAT}RIX \textbf{LAB}ORATORY ed è specializzato nella manipolazione di matrici di grandi dimensioni come quelle del nostro progetto. 
        
        Per la risoluzione di sistemi lineari, MATLAB mette a disposizione diverse funzioni, le più conosciute sono \textit{chol}~\cite{cholmatlab},  \textit{mldivide}~\cite{mldividematlab} e \textit{decomposition}~\cite{decompositionmatlab}. La prima funzione analizzata è stata \textit{chol} ma si è rivelata molto lenta nell'esecuzione. Si è quindi deciso di passare a \textit{mldivide} ma per come è strutturata la funzione non c'è modo di essere sicuri che il metodo utilizzato sia la decomposizione di Cholesky in quanto la funzione è pensata per risolvere molti tipi di sistemi lineari. La scelta finale è stata quella di utilizzare la funzione \textit{\textbf{dA = decomposition(A,type,triangularFlag)}}. Questa funzione ha garantito ottime prestazioni e la certezza di utilizzare il metodo di Cholesky.\\[0.5cm]
        
        \lstinputlisting[language=Matlab,label={matlabcode}, caption={Codice MATLAB}]{src/code/matlab.m}
        
        \vspace{0.5cm}
    	
    	\noindent Per misurare i tempi di esecuzione e la memoria utilizzata abbiamo utilizzato un \textit{profiler}~\cite{profilermatlab} che permette di visualizzare con precisione i dati che servivano per l'analisi dei risultati, permettendo di prendere in considerazione soltanto la parte di codice che serviva.
        
        \newpage
        \subsection{OCTAVE}
        \noindent Primo ambiente scelto per effettuare un confronto in quanto è l'ambiente più vicino a MATLAB. I due ambienti hanno praticamente le stesse funzionalità e accettano lo stesso tipo di codice. \href{https://www.gnu.org/software/octave/}{OCTAVE}~\cite{octave} è un ambiente open-source distribuito con licenza \textit{GNU GPL v3}.
        Nonostante sia \textit{"la versione gratuita di MATLAB"} esso non offre proprio tutte le stesse funzionalità, per cominciare non è presente la funzione \textit{decomposition}. Esistono però le controparti \textit{chol}~\cite{choloctave} e \textit{mldivide}~\cite{mldivideoctave}. Nel primo caso i test e i risultati si sono rilevati molto scadenti in termini di utilizzo della memoria e tempi di calcolo, in quanto utilizzando questa funzione non era in grado di ritornare un risultato neanche per matrici di dimensioni medie (es. apache2 e parabolic\_fem), abbiamo deciso quindi di utilizzare la sua alternativa. Il metodo \textit{mldivide} ha invece ritornato buoni risultati per quanto riguarda l'utilizzo della memoria e dei tempi di calcolo.\\[0.5cm]
        
        \lstinputlisting[language=Octave,label={octavecode}, caption={Codice Octave}]{src/code/octave.m}
        
        \vspace{0.5cm}
        
        \noindent Per quanto riguarda la misurazione dei tempi di calcolo e memoria  utilizzata il profiler offerto da OCTAVE non permette la visualizzazione di quest'ultima è stato quindi necessario effettuare misurazioni \textit{"approssimative"} attraverso l'utilizzo dei strumenti esterni offerti dai sistemi operativi.
        
        \newpage
        \subsection{PYTHON}
        \noindent Ambiente scelto per la sua versatilità e nonostante non sia specializzato nell'elaborazione nel calcolo scientifico permette lo sviluppo di programmi più complessi e completi.\\
        Per raggiungere il nostro scopo abbiamo scelto di utilizzare la libreria \textit{scikit-sparse}~\cite{scikit-sparse} che contiene un wrapper per la libreria CHOLMOD della suite \textit{suiteparse}~\cite{suitesparse}, questo consente quindi di applicare la decomposizione di Cholesky.
        
        La libreria ha restituito buoni risultati prestazionali in termini di memoria utilizzata e tempi di calcolo ma a causa di problematiche relative alla sua compatibilità con Windows nonè stato possibile effettuare un confronto di prestazione in OS diversi.
        Esistono dei package~\cite{package} per l'istallazione automatica della libreria su Windows disponibili su GitHub ma come descritto nelle documentazioni non assicurano sempre una corretta installazione non sempre rendono disponibili tutte le funzionalità della libreria, quest'ultimo si tratta del nostro caso.
        Per questo motivo abbiamo deciso di non effettuare il confronto tra ambiente Linux e ambiente Windows, questa decisione ha influito sulle considerazioni finali della libreria. 
        
        Non sono state prese in considerazioni altre librerie di Python in quanto durante la fase di test si sono rivelate molto inefficienti in termini di memoria e tempi di esecuzione, per questo motivo sono state scartate.\\[0.5cm]
        
        \lstinputlisting[language=Python,label={pythoncode}, caption={Codice Python}]{src/code/python.py}
        
        \vspace{0.5cm}
        
        \noindent Infine abbiamo utilizzato la libreria \textit{time} per calcolare i tempi di calcolo e la libreria \textit{memory-profiler}~\cite{profilerpython} per monitorare l'utilizzo di memoria da parte di una singola funzione
    
    \newpage
    \section{Ambiente di Benchmark}
    La macchina utilizzata per risolvere i sitemi è un \textbf{Satellite L50-A-1D6} con le seguenti caratteristiche:
    \begin{itemize}
      \item \textbf{Processore}: i7-4700MQ
      \item \textbf{RAM}: 8 GB DDR3
      \item \textbf{HDD}: 1TB SSD
      \item \textbf{Versione Windows}: Window 10 Pro
      \item \textbf{Versione Ubuntu}: Ubuntu 18.04
    \end{itemize}
    
    \section{Metrice per il Confronto}
    \noindent Come indicato nella consegna del progetto i paragoni oggettivi tra i vari ambienti andavo eseguiti su 3 fronti, \textbf{Tempo di Esecuzione}, \textbf{Memoria Occupata} e \textbf{Errore Relativo}.
    \begin{itemize}
        \item \textbf{Tempo di Esecuzione:} tempo impiegato per la risoluzione dei sistemi, calcolata in secondi.
        
        \item \textbf{Memoria Occupata:} memoria utilizzata per eseguire le funzioni, calcolata in MB e convertita per comodità in GB.
        
        \item \textbf{Errore Relativo:} errore relativo tra soluzione ritornata dal calcolo \textit{x} e dalla soluzione esatta \textit{xe}.\\
        \begin{equation}
        \begin{aligned}
            error = \frac{\|x - xe\|_2}{\|xe\|_2}\\
            \text{dove $\|v\|_2$ è la norma euclidea del lettore v}.
        \end{aligned}
        \end{equation}
         
    \end{itemize}\\[0.5cm]
    Altre metriche come come la facilità d'uso, la documentazione a disposizione e se esse siano ancora mantenute dagli sviluppatori sono state prese in considerazione durante l'analisi delle librerie. 
    
\chapter{Analisi dei Risultati}
    \section{Cofronto MATLAB tra OS}
    \noindent Come si può notare dal grafico in figura \ref{plot:timemat} l'elaborazione in ambiente Linux risulta leggermente più veloce rispetto a quella in Windows.
    \begin{figure}[ht]
        \centering
        \includegraphics[scale=0.8]{src/plot/TimeMAT.png}
        \caption{Tempi di esecuzione a confronto}
        \label{plot:timemat}
    \end{figure}
    
    \noindent Per quanto riguarda la \textbf{memoria occupata} abbiamo notato, anche in questo caso, come Linux sia in grado di essere migliore di Windows, nonostante si possa notare che per la maggior parte dei casi le differenze sono infinitesimali.\\[0.5cm]
    
    \begin{figure}[ht]
        \centering
        \includegraphics[scale=0.8]{src/plot/MemMAT.png}
        \caption{Memoria occupata a confronto}
        \label{plot:memmat}
    \end{figure}
    
    \newpage
    \noindent Infine come si può notare in figura \ref{plot:memmat} in generale non esistono differenze rilevanti tra i due ambienti.\\[0.5cm]
    
    \begin{figure}[ht]
        \centering
        \includegraphics[scale=0.8]{src/plot/ErrMAT.png}
        \caption{Errori Relativi a Confronto}
        \label{plot:errmat}
    \end{figure}
    
    \newpage
    \section{Confronto OCTAVE tra OS}
    \noindent In figura \ref{plot:timeoct} si può notare come i \textbf{tempi di esecuzione} siano in genere a favore di un sistema operativo Linux.
    
    \begin{figure}[ht]
        \centering
        \includegraphics[scale=0.8]{src/plot/TimeOCT.png}
        \caption{Tempi di esecuzione a confronto}
        \label{plot:timeoct}
    \end{figure}
    
    \noindent Per quanto riguarda la \textbf{memoria occupata}, come si può notare in figura \ref{plot:memoct}, esecuzione in ambiente Linux si è rivelata più performante rispetto a Windows, anche se non con una differenza significativa.
    
    \begin{figure}[ht]
        \centering
        \includegraphics[scale=0.8]{src/plot/MemOCT.png}
        \caption{Memoria occupata a confronto}
        \label{plot:memoct}
    \end{figure}
    
    \noindent Infine per quanto riguarda l'\textbf{errore relativo} non sono state riscontrate differenze significative, tenendo conto dell'ordine di grandezza alla quale facciamo riferimento.
    
    \begin{figure}[ht]
        \centering
        \includegraphics[scale=0.8]{src/plot/ErrOCT.png}
        \caption{Errori Relativi a Confronto}
        \label{plot:erroct}
    \end{figure}
    
    \section{Confronto PYTHON tra OS}
    \noindent Non è stato possibile effettuare un confronto tra Windows e Python con la libreria scelta a causa di alcuni problemi noti agli sviluppatori durante l'installazione nel sistema operativo Microsoft.

    \newpage
    \section{Confronto tra Librerie}
        \subsection{Windows}
        
        \noindent Come già specificato nei capitoli precedenti i confronti in Windows risultano soltanto due, MATLAB e OCTAVE, in quanto non è stato possibile completare una installazione corretta della libreria PYTHON.\\
        Come si può notare dalla figura \ref{plot:timewin} OCTAVE risulta sempre più lento nell'esecuzione di tutte le matrici arrivando a raggiungere addirittura una differenza di 20 secondi sulla matrice \textit{G3\_circuit}.\\
        
        \begin{figure}[ht]
            \centering
            \includegraphics[scale=0.8]{src/plot/WINtime.png}
            \caption{Tempi di esecuzione a confronto}
            \label{plot:timewin}
        \end{figure}
        
        \noindent Per quanto riguarda la \textbf{memoria occupata} se si considera il grafico \ref{plot:memwin} OCTAVE sembrerebbe ancora una volta avere la peggio su MATLAB. Questo dato però risulta falsato in quanto il calcolo eseguito con MATLAB non tiene conto della memoria utilizzata dall'interfaccia del IDE a differenza di OCTAVE dove, non potendo eseguire un calcolo preciso della memoria occupata, si è dovuto prendere in considerazione tutta la memoria occupata dal programma.\\
        Se si considerano  questi fatti OCTAVE risulta occupare meno memoria rispetto alla sua controparte.\\
        
        \begin{figure}[ht]
            \centering
            \includegraphics[scale=0.8]{src/plot/WINmem.png}
            \caption{Memoria occupata a confronto}
            \label{plot:memwin}
        \end{figure}
        
        \newpage
        \noindent Per quanto riguarda l'\textbf{errore relativo}, non sono state riscontrate differenze sostanziali tra le due librerie.
        
        \begin{figure}[ht]
            \centering
            \includegraphics[scale=0.8]{src/plot/WINerr.png}
            \caption{Errori Relativi a Confronto}
            \label{plot:errwin}
        \end{figure}
        
        \newpage
        \subsection{Linux}
        
        \noindent A differenza di Windows, siamo riusciti a fare anche un paragone con la libreria di PYTHON\\
        Quest'ultima si è rilevata meno performante in termini di \textbf{tempi di esecuzione}, come si può notare dal grafico \ref{plot:timelin}. Sempre dal grafico possiamo evincere come MATLAB sia quasi sempre (ad esclusione di un caso) la soluzione migliore.\\
        
        \begin{figure}[ht]
            \centering
            \includegraphics[scale=0.8]{src/plot/LINUXtime.png}
            \caption{Tempi di esecuzione a confronto}
            \label{plot:timelin}
        \end{figure}
        
        \noindent Per quanto riguarda la \textbf{memoria occupata}, come si può notare in figura \ref{plot:memlin} PYTHON risulta avere un'andamento più lineare rispetto ai suoi "rivali", mentre, sempre considerando la differenza nel reperire le informazioni tra MATLAB e OCTAVE, possiamo notare non siano presenti differenti sostaziali, da questo ne deduciamo che, aggiungendo anche il peso del interfaccia di MATLAB, quest'ultimo risulti il peggiore in questo confronto.\\
        
        \begin{figure}[ht]
            \centering
            \includegraphics[scale=0.8]{src/plot/LINUXmem.png}
            \caption{Memoria occupata a confronto}
            \label{plot:memlin}
        \end{figure}
        
        \newpage
        \noindent Infine, anche per quest'ultimo confronto sull'\textit{errore relativo} non sono state riscontrate differenze sostanziali nei risultati.\\
        
        \begin{figure}[ht]
            \centering
            \includegraphics[scale=0.8]{src/plot/LINUXerr.png}
            \caption{Errori Relativi a Confronto}
            \label{plot:errlin}
        \end{figure}

\newpage
\chapter{Conclusioni}

\noindent Prima di trarre delle conclusioni riteniamo corretto fare alcune precisazioni.\\
La prima riguarda i risultati del calcolo degli errori relativi in tutti gli ambienti. Come si può notare da tutti i grafici che riguardano questo argomento, l'errore relativo risulta avere valori molto elevati in una matrice in particolare, \textit{ex15}, questa matrice risulta essere anche la più piccola. Questo risultato potrebbe sembrare sbagliato e insensato in una prima e veloce analisi ma questo fenomeno è spiegato dal \textbf{condizionamento}~\cite{cond} di una matrice. Questa quantità, sempre maggiore di 1, indica la difficoltà della risoluzione di un sistema lineare, più questo valore si avvicina a 1 più è facile la sua risoluzione. La matrice \textit{ex15} risulta avere un condizionamento molto alto e ciò può portare ad un errore relativo più elevato.\\
Bisogna inoltre precisare che la libreria migliore in assoluto probabilmente non esiste. Le librerie vanno scelte di volta in volta a seconda dei risultati che si vogliono ottenere e dalla disponibilità economica e di hardware. Ad esempio se l'obbiettivo è eseguire operazioni che rientrano esclusivamente nel calcolo scientifico, servono buone prestazioni ed il budget è sufficiente la scelta consigliata è MATLAB. Se la risoluzione del sistema fa parte di un sistema software più complesso con magari la necessità di usare API o svolgere anche altri tipi di funzioni probabilmente la scelta potrebbe ricadere su PYTHON. Ancora, se la macchina a nostra disposizione non possiede molta memoria (RAM) ed il budget è ristretto la scelta potrebbe ricadere su OCTAVE. Questi esempi danno un'idea delle variabili che potrebbero influire sulla scelta da effettuare. Grazie ai dati forniti da questo progetto l'obiettivo è consentire a chi vuol scegliere una libreria di avere più dati possibili per effettuare la scelta migliore.
    
    \newpage
    \section{Sintesi}
    MATLAB si è rilevato in generale il migliore in termini di \textbf{tempi di esecuzione} a discapito della \textbf{memoria utilizzata} durante l'esecuzione.
    Il dettaglio più importante di MATLAB è che si tratta di un software proprietario le cui licenze possono risultare onerose per una persona sola.
    
    Al contrario OCTAVE ha una licenza \textit{Open Source}  ma risulta più lento e presenta una documentazione di qualità inferiore rispetto a MATLAB.
    
    PHYTHON è gratuito, ben documentato,ha una community molto attiva e numerosa dalla quale si riescono spesso a trovare le soluzioni ai problemi che si riscontrano nel corso della creazione del software. Inoltre è migliore in caso si desideri creare una applicazione più complessa che vada oltre la risoluzione del problema matematico ed integri i risultati per implementare funzionalità avanzate. Il principale problema di PYTHON è il gestire la compatibilità da una versione all'altra e cross-platform. Il problema di compatibilità è presente anche in MATLAB ed OCTAVE ma una forma meno critica.
    
    \begin{table}[h!]
    \centering
    \begin{tabular}{|| c || c c c c ||}
    \hline
    Ambiente & Performance & Memoria & Documentazione & Costi \\ [0.5ex]
    \hline
    \hline
    \textbf{MATLAB} & X &  & X & \textcolor{red}{X} \\
    \textbf{OCTAVE} &   & X &  &  \\
    \textbf{PYTHON} &  & X & X &  \\ [1ex]
    \hline
    \end{tabular}
    \caption{Tabella riassuntiva dei Risultati}
    \end{table}
    
    \noindent Come ultima nota, nessuna delle due macchine disponibili è stata in grado di elaborare le due matrici più impegnative: \textit{Flan\_1565} e \textit{StocF-1465}. Queste due matrici nella loro complessità avrebbero potuto giocare un ruolo importante nella scelta della libreria migliore ma l'hardware non era adeguato e ciò potrebbe avere influito sulle considerazioni finali.
    
\nocite{*}
\printbibliography

\end{document}